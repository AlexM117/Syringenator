University of Washington

T\+C\+ES 460

Winter 2019\hypertarget{index_autotoc_md1}{}\subsection{Development Team Hephaestus}\label{index_autotoc_md1}
\begin{DoxyAuthor}{Authors}

\end{DoxyAuthor}
\tabulinesep=1mm
\begin{longtabu}spread 0pt [c]{*{2}{|X[-1]}|}
\hline
\PBS\centering \cellcolor{\tableheadbgcolor}\textbf{ Name  }&\PBS\centering \cellcolor{\tableheadbgcolor}\textbf{ Email   }\\\cline{1-2}
\endfirsthead
\hline
\endfoot
\hline
\PBS\centering \cellcolor{\tableheadbgcolor}\textbf{ Name  }&\PBS\centering \cellcolor{\tableheadbgcolor}\textbf{ Email   }\\\cline{1-2}
\endhead
Alex Boyle  &???   \\\cline{1-2}
Ammon Dodson  &\href{mailto:ammon0@uw.edu}{\texttt{ ammon0@uw.\+edu}}   \\\cline{1-2}
Alex Marlow  &\href{mailto:alexmarlow117@gmail.com}{\texttt{ alexmarlow117@gmail.\+com}}   \\\cline{1-2}
Jake Mc\+Kenzie  &\href{mailto:jake314@uw.edu}{\texttt{ jake314@uw.\+edu}}   \\\cline{1-2}
Brooke Stevenson  &???   \\\cline{1-2}
\end{longtabu}


\begin{DoxyCopyright}{Copyright}
Copyright \copyright{} 2019 by the authors. All rights reserved.
\end{DoxyCopyright}
\hypertarget{index_autotoc_md2}{}\subsection{Hypo\+Robot Assignment}\label{index_autotoc_md2}
If you’ve been paying any attention at all to current events you know that a major plague has descended on cities and counties throughout the country in the form of used and discarded hypodermic needles. Countless hours are spent cleaning up this mess. For instance, some schools are forced, for safety reasons, to send staff out to scour the playgrounds prior to children showing up.

Your task this quarter will be to design an autonomous robot that can help automate the arduous and sometimes dangerous job of spotting, retrieving, and disposing of hypodermic syringes.

Your robot will be a prototype, not a fully functional disposal robot, but it will have important technical features necessary on such a robot.

A second point is that we will be dealing with industrial (i.\+e. dull) syringes. These are typically used to disburse such things glue or solvents. They are commonly used in our labs to glue acrylic parts together. Anyone in the lab with a sharp needle will be immediately disqualified. Even so, if you would rather not design and test with any syringe, you may, with my written permission, use a ballpoint pen, a \#2 pencil or a similar object of your choosing.

All testing will be done indoors on a flat surface.

--Robert Gutmann, Ph.\+D. 